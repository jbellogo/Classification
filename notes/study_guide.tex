\documentclass{article}

\usepackage{amsfonts}

\usepackage{graphicx}
\graphicspath{ {./images/} }

\usepackage{bm}

\usepackage{float}
\newtheorem{theorem}{Theorem}





\usepackage{amsmath}

\usepackage[a4paper, total={6in, 8in}]{geometry}


\setlength{\parindent}{0pt} % for getting rid of first line indentation





% ================================================
% =====================  BEGIN DOCUMENT 
% ================================================

\title{STAT 441 Study}
\author{Juan Pablo Bello Gonzalez}
\date{September 2024}

\begin{document}

\maketitle

\newpage

\section{1. Learning Concepts}
Learning:
\begin{itemize}
\item Classification: Categorical outcome
\item Regression: Continuous outcome
\end{itemize}

Supervised: You have labels, ie right answers/confirmation for training data. Easier models than unsupervised. In unsupervised you have to make up distinctive features/categories, you don't even know how many clusters there are.

\textbf{Interpretation vs. prediction tradeoff}:
Complex models like trees and neural netoworks are powerful at predicting but there is little interpretability to what the model coefficients actually mean. 

You can restrict the class of models to allow for easy interpretation, ie trees with tunned hyperparameters. THEN optimize for prediction wihtin this class models. 
\subsection{Loss}
MSE

\subsection{Bias-Variance Tradeoff}
MSE loss can be decomposed into three sources of error: \textbf{Bias, Variance, and Irreducible error} 
\[ E [ (y_0 - \hat f(x_0))^2]
= Var(\hat f(x_0))) + [Bias(\hat f(x_0)))]^2 + Var(\epsilon)
\]
All three are non-negarive. 
\begin{itemize}
\item Variance: How well the model generalizes to other data sets. 
\item Bias: how well the model fits the training set.
\end{itemize}

Overfitting: low bias high variance. Perfect fit  from highly flexible model, usually with many hyperparameters, means that the model will have high variance and not predict other test sets as well as it predicts that one. 

Explain the tradeoff:
- Conflict of trying to simulataneously minimize the two sources of error that prevent supervised learning algorithms from generalizing beyond the training set. 

MSE is the sum of three things, the graph of MSE makes a U, and the graphs of its three components are
\begin{itemize}
\item Irreducible error/Var(epsilon). Constant
\item Variance. Increasing with flexibility. The more flexibility, the worse predictive power to  data sets outside the training set the model will have.
\item Bias. Decaying as flexibility increases. Higher flexibility $\implies$ lower bias $\implies$  better fit. 
\end{itemize}

\subsection{Bayes Classifier}
A classification algorithm takes in a unit with features (a row on a table) and gives a probability distribution over the categories. Baye's classifier says, classify the unit with the most likely class.

\begin{theorem}
\textbf{Conditional Bayes Error Rate}: For a given obervation $x$, is $1-max_{j} P(Y =j |x)$ the complement of the Bayes classifier probability. 
\end{theorem}

\begin{theorem}
\textbf{Overall Bayes Error Rate}:  $1-E[max_{j} P(Y =j |x)]$ the EXPECTATION complement of the Bayes classifier probability. 
\end{theorem}

\section{2. Practical aspects}

\subsection{overfitting}
\textbf{Definition:} the model fits the \textbf{random noise} of a \textbf{sample} rather than the generalizable relationship.

Occurs when the model is \textbf{too flexible}. Has too many parameters relative to the number of observations. Advanced learning algorithms are flexible, you don't need a neural network to predict y based on two features, just use some regression.

\subsubsection{Defending against overfitting}
\begin{itemize}
\item Fit to training set
\item Evaluate on test set
\item split must be chosen AT RANDOM. but fix seed in practice.
\end{itemize}


By splitting the data into test/train, overfitting can be avoided. If you train on 100\% of the data there will be overfitting!
We build a model based on the training data but evaluate the model on the test data. 

\subsection{cross-validation}
Split data into k random subsets of equal size, called folds, then fit a model to each of the possible combinations of k-1 folds and evaluate/test on the remaining fold. Average accuracty scores to obtain a good estimate of the true predictive power of the model.

In extreme case: leave-one-out (LOO) cross validation. It's nonsense for most applications.

\subsection{Evaluation measures}
\begin{itemize}
\item Accuracy
\item Sensitivity 
\item Specificity
\item Area under the ROC curve
\item F
\end{itemize}

for classification, consider: 
\begin{itemize}
\item TP 
\item FP
\item RN
\item FN
\end{itemize}

the ordering in the name is (Correct label?, model classification label). Correct label? means that those that start with a true: TP, TN were correctly classified. The seonc instance is what that label classification was duh.

\textbf{Confusion Matrix}
\[
\begin{array}{|c|c|c|}
\hline
 & \text{Predicted Positive} & \text{Predicted Negative} \\
\hline
\text{Actual Positive} & TP & FN \\
\hline
\text{Actual Negative} & FP & TN \\
\hline
\end{array}
\]

\subsubsection{Accuracy, Sensitivty, and Specificity}
some formulas for evaluation measures
\begin{itemize}
\item Accuracy: (TP+TN)/N, where N is the sum of all four = \#observations or sample size. Also Accuracy = $\frac{1}{n} \sum_{i=1}^n I(y_i = \hat y_i)$
\item Sensitivity (TRUE POS rate): TP/(TP + FN)
\item Specificity (TRUE NEG rate): TN/(TN+ FP)
\end{itemize}

Sensitivity is positive. 
$1-\text{Specificity}$ is False positive rate. 

Ideally we want high specificty and sensitivity.But a trade off needs to be made. The threshold of prediction... you may want to be conservative in your true predictions (high specificity low sensitiviy) if it is for Cancer diagnosis. But liberal if it is about who gets on a survival boat.

\subsubsection{ROC}
Area under curve is a measure of the sensitivity/specificity tradeoff. 

\subsubsection{F-measure}
\[
F = \frac{2 \times TP}{2 \times TP + FP +FN}
\]

\begin{itemize}
\item Higher values are better. 
\item appropiate for 0/1 classification.
\end{itemize}

Important: 
- Sometimes, depending on the context, you may want to always predict false. An example: when there are ratre or uncommon events. 
- F measure is affected the most by this. Even if you don't predict false all the time, as long as there is a clear bias in the distribution of false classifications, your TP rate will be very low and so your F-measure will be small as well. 


\subsubsection{Dealing with rarely occuring categories}
Not uncommon to have rare categories, ie. categories that occur infrequently.
\begin{itemize}
\item Macro averaging: Treat all categories the same. Default of all algorithms. Default of not doing anything in preprocessing.
\item \textbf{MICRO} averaging: dominate by frequent categories, giving little weight to rare categories.
\end{itemize}


\subsubsection{One-hot-encoding or "factoring" in R}
Sometimes, variates are given in terms of categories themselves like X2 = "Family role" $\in$ [1 : "Father", 2: "son", 3:"sibling"]. \\

We shouldn't just feed these categories as they are to an algorithm because the scale and ordering provided by the category numbers, $[1,2,3]$ in the prior example, doesn't mean anything real, and could at best add noise to the data, and at worse confoundingly influence the predictions.\\ 

So for a categorical variate with $n$ categories, we take $n-1$ indicator variables to represent the presence of that category in that variate, this is sometimes known as factoring. \\

why not do $n$ indicators, one for each category number? Because that would cause multicolinearity. As the $n-$th indicator is just the absence of the $n-1$ other ones (you can write them in a linear dependence)

\textbf{NOTE:} We often don't leave-one-out in advanced statistical models because they don't always use all variates, or collinearity doesn't affect the model like in regression, where you absoluteley cannot have dependencies between variates.


\subsubsection{Variable scaling}
Do when variables have vastly differnt ranges, helps give bettwr results by making variates more comparable to each other. 
Scaling is also called standardization.

Scaling methods:
\begin{itemize}
\item Scale them to have mean zero and stdev =1 using $x_{scaled} = \frac{x-\bar x}{sd_x}$. Note that indicator variables are not scaled like this, their mean and variance may not be 0,1.
\item Scale variables to have the same range.
\end{itemize}

\section{Logistic Regression}
Appropiate for 2 class classification $0,1$.
\begin{itemize}
\item $P(y==1) =p$
\item $E(y) = p$
\end{itemize}

We are using a multilinear regression on all the variates (you can do polynomial thourugh preprocessing by say adding columns $x_2^3$ to the dataframe)

\[
\text{logit}(p) = \text{log} (\frac{p}{1-p}) = X \bold\beta 
\]

\textbf{logit} tips:
\begin{itemize}
\item Recall that log($p$) is defined on $x>0$. So logit($p$) is only ever defined on $p>1-p := p \in (0,1)$
\item logit has Range = $(-\infty, \infty)$ and domain $(0,1)$, so what it does, implicitly, is take values $p$ from $[0,1]$ onto the real line.
\end{itemize}

To \textbf{recover} the succcess/label 1 probability, p, we use the \textbf{expit} function:
\[ expit(x) = \frac{\exp(x)}{1+exp(x)}\]
\[ p = text{expit}(\beta_0 + \beta_1x) = \frac{\exp(\beta_0 + \beta_1x)}{1+\exp(\beta_0 + \beta_1x)}\]

As you would expect, p is equal to a function that takes the linear model onto the range $0,1$ which is fit for a probability. 

\subsubsection{Odds}
\[odds = \frac{p}{1-p}\]
log odds is exactly what it sounds. 

\textbf{A coefficient can be interpreted as a log odds ratio of a one unit differnce in the direction of its $x$ variable:}
If

$logit(p_i) = \beta_0 + \beta_1x + \dots + \beta_p x_p$

and 
$logit(p_j) = \beta_0 + \beta_1(x+1) + \dots + \beta_p x_p$
Then $logit(p_j)-logit(p_i) = \beta_1$

\[
\beta_1 = log(\frac{odds_j}{odds_i})
\]

For interpretation:
Exponential coefficients give the log odds ratio of one unit increase in the x variate of the coeficient.

So if $logit(p_{admission}) = 10+.15x_{average}$

you can say: "on average, each 1\% increase in average \textbf{increases the odds} of admission by \textbf{a factor of} $exp(\beta_1) = exp(0.15)$, with all other variables held constant."

So if the odds ($\frac{p}{1-p}$) of admission for an average of $x_1 = 80$ are, say 7.38, then the revised odds for an average of $81$ are = $7.38 \times exp(.15)$

WE can only make statements about odds. We cannot say anything about how the probabilities $p$ themselves will change when you increase one variate by 1 because the increase is non-constant. 

\subsubsection{model evaluation}
Logistic regression is less prone to overfitting than other learning methods.
\begin{itemize}
\item Linearity assumption makes the model highly inflexible.
\item There is no tunning parameter by default... you could regularize.
\item Usually no test/train/cv split for logistic regression needed.
\item it is hard to beat logistic regression (bi class) in terms of prediction. 
\item it is easier to beat linear regression or multi-class logisti regression.
\end{itemize}

\subsubsection{Shaky}
the log likelihood maximization and matrix terms.


\section{Regularization}
\begin{itemize}
\item L1: Lasso
\item L2: Ridge
\end{itemize}

\subsubsection{Motivation}
In regression, OLS is the 'best' (lowest variance) \textbf{unbiased} estimator. 
However, we may be willing to trade a bit of bias for a larger reduction in variance (better prediction)

Recall the bias variance tradeoff:  bias and variance cannot be simultaneously minimized.

As model felxibility/complexity/tunnability/\# hyperparameters increases:
\begin{itemize}
\item Bias decreases: Better fit to the training set. 
\item Variance increases: Less generalizability to other samples of the population.
\end{itemize}

For low flexibility (\textbf{UNDERFITTING}:) High bias, low variance. For high flexibility, \textbf{OVERFITTING}, low bias, high variancce. 

Regularization explicitly addresses the tradeoff between overfitting and underfitting. Overfitting occurs when a model is overly flexible, characteristic of highly nonlinear and complex models. However, we also want to avoid underfitting, as using complex models can help capture the true relationships between labels and variables. To balance these opposing goals, regularization allows us to use complex models while adding a penalty for model complexity. This ensures that, among a set of complex models, we select the one that is sufficiently complex without being excessively so.

the new criterion to minimize becomes $MSE + Penalty(\beta)/n = RSS + Penalty(\beta)$
\[ \min_{\bm \beta} \{ RSS + penalty \}\]

where $RSS = \sum_{i=1}^n (y_i- \beta^t x)^2$

\subsubsection{Ridge, L2}

\[ Penalty(\beta) = ||\beta|| = \sqrt{\sum_{i=1}^p \beta_i^2}\]
Note, we don't penalize the intercept $\beta_0$, i.e. it is not included in the computation of the penalty. You can see this in the fact that the summation in ridge starts at $i=1$ and not at 0.

For minimization purposes, get rid of the square root to simplify things. \\
\[ \text{Criterion} = 
\frac{1}{n}
( RSS + \sum_{i=1}^p \beta_i^2 )
\]

Now, we introduce a penalty as a function of the coefficients, but we can also include a parameter, known as the \textbf{regularization parameter} $\lambda$, to adjust its strength. This parameter allows us to fine-tune the penalty, as some datasets may require a stronger penalty than others.

\[ \text{CriterionL2} = 
\frac{1}{n}
( RSS +  \lambda \sum_{i=1}^p \beta_i^2 )
\]

\begin{enumerate}
\item Tunning: estimate lambda via CV (training on multiple sets)
\item Training: find the beta's in the model that minimize the criterion, with the tuned parameter.
\end{enumerate}

\textbf{Notes:}
\begin{itemize}
\item The penalty is a function of squared coefficients, so it penalizes larger coefficients more than smaller coefficients. 
\end{itemize}

if you don't want to penalize larger coefficients as strongly, you can use an L1 penalty, which penalizes them lineary. As a bonus, L1 penalty also performs variable selection for ya. 

\[ \text{CriterionL1} = 
\frac{1}{n}
( RSS +  \lambda \sum_{i=1}^p |\beta_i| )
\]

\subsubsection{L1 vs L2}
\begin{itemize}
\item Usually L2 works better in terms of accuracy/MSE.
\item L1 beats L2 when it does variable selection, i.e. when there are a lot of \textbf{irrelavant} variables in the data. 
\item Coefficients in L1 can shrink to zero. Coefficients in L2 do not shrink all the way to zero. Therefore, L1 is used for var selection.
\item L1 pernalty is numerically harder to compute, no clean derivatives of absolute values. 
\end{itemize}

Recall that the abstraction of RSS, its generalized  criterior to optimize is \textbf{likelihood}. We can add penalties to likelihiod to add regularization to generalized models like logistic regression.

For L2 penalty, in general:

\[Criterion =  -loglikelihood + \frac{\lambda}{n} \sum_{i=1}^p \beta_j^2\]

But in practice, we often scaale the squred coefficients in the penalty by the variance of the variate. 
\[Criterion =  -loglikelihood + \frac{\lambda}{n} \sum_{i=1}^p \beta_j^2 Var(x_j)\]


\section{Text}
























\end{document}
